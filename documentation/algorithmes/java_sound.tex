% Yvonnick Esnault - yvonnick.e@free.fr 
% Gaetan Le Brun - gaetan.lebrun@gmail.com
% Thibaut Leli�vre - thibaut.lelievre@voila.fr
% Vincent Mah� - vmahe@free.fr
% Les chapitres peuvent commencer sur une page paire ou impaire
%openany
\documentclass[a4paper,11pt]{article}
%\frontmatter

\usepackage[T1]{fontenc}
\usepackage{ae,aecompl}
\usepackage[frenchb]{babel}
\usepackage[pdftex]{graphicx}
\usepackage[french]{minitoc}
\usepackage{fancyhdr}
\usepackage{lastpage}
\usepackage{hyperref}
\usepackage{textcomp}
%\usepackage [applemac] {inputenc} % les accents
\usepackage{verbatim}
\usepackage{float}

\evensidemargin=13pt
\oddsidemargin=13pt
\topmargin=-\headheight \advance\topmargin by -\headsep
\textwidth=14.59cm \textheight=21.62cm % normal A4, 1'' margin % 24.62cm
\setlength{\parindent}{8mm} % on remet le retrait de paragraphe

%%% My Vars %%%
\def\mytitle{TetraHead}
\def\mysubject{Synth\'etiseur}
\def\mydate{\today}
\def\myversion{1.0}
\def\me{\small  Yvonnick Esnault - Ga�tan Le Brun - Thibaut Leli�vre - Vincent Mah�}
\def\meandmail{\me \\{\normalsize\tt \mymail}}

%%% Fancy headers to add logo, line \ldots{} %%%
\pagestyle{fancy}
\fancyhf{}

% red�finition du style plain pour les pages sp�ciales (chapter)
\fancypagestyle{plain}{
 \fancyhead{}
 \renewcommand{\headrulewidth}{0pt}
 \renewcommand{\footrulewidth}{0.4pt}
 \lfoot{\textsl{\me\space} \\
\mysubject}
 \cfoot{}
 \rfoot{\textsl{Page \thepage \space sur \pageref{LastPage} }}
}

\renewcommand{\headsep}{50pt}

%\renewcommand{\chaptermark}[1]{\markboth{#1}{}}
\renewcommand{\sectionmark}[1]{\markright{#1}}
%\lhead[\thepage]{\rightmark}
%\rhead[\leftmark]{\thepage}

% Header
%\lhead{\includegraphics[width=2cm]{../document_type/images/logo.png}}
\chead{\mysubject} %\leftmark : contient le nom du chapitre courant.
%\rhead{\includegraphics[width=2cm]{images/logo_itin.png}}

%Footer
\lfoot{\textsl{\me\space}}
\cfoot{}
\rfoot{\textsl{Page \thepage \space sur \pageref{LastPage} }}

%Affiche les ligne en haut, et en bas
\renewcommand{\headrulewidth}{0.4pt}
\renewcommand{\footrulewidth}{0.4pt}

%%% bookmarks, linking in pdf %%%
\hypersetup{pdfauthor={\me},%
            pdftitle={\mytitle},%
            pdfsubject={\mysubject},%
            colorlinks,%
            citecolor=black,%
            filecolor=black,%
            linkcolor=black,%
            urlcolor=black,%
            pdftex}

%%% Commande pour supprimer en-t�tes
%%% et pied de page des pages vides
\newcommand{\clearemptydoublepage}{%
\newpage{\pagestyle{empty}\cleardoublepage}}

%%% Environnement pour le tableau de l'historique
\newenvironment{historique}
{

\begin{center}
{\huge \textbf{Historique}}

\vspace{3cm}
\begin{tabular}{|l|l|l|l|}
\hline
\textbf{Date} & \textbf{Auteur} & \textbf{Modifications} & \textbf{Version} \\
\hline
}
{
\end{tabular}\end{center}
\newpage
}

\newcommand{\histo}[4]
{#1 & #2 & #3 & #4 \\
\hline
}


%\newcommand{\info}[1]{
%\begin{figure}
%  \framebox[\textwidth][l]{
%   \begin{minipage}[c]{0.1\textwidth}
%     \medskip
%
%      \includegraphics{../document_type/images/info.png}
%      \medskip
%
%   \end{minipage} \hfill
%   \begin{minipage}[c]{0.8\textwidth}
%     \medskip
%
%     #1\medskip
%
%   \end{minipage} \hfill
%   \begin{minipage}[c]{0.1\textwidth}
%     \makebox[30pt]{}\par
%   \end{minipage}
%}
%\end{figure}
%}

\newcommand{\encadre}[2]{
\begin{figure}[!h]
  \framebox[\textwidth][l]{
   \begin{minipage}[c]{0.1\textwidth}
     \medskip

      \includegraphics{#1}
      \medskip

   \end{minipage} \hfill
   \begin{minipage}[c]{0.8\textwidth}
     \medskip

     #2\medskip

   \end{minipage} \hfill
   \begin{minipage}[c]{0.1\textwidth}
     \makebox[30pt]{}\par
   \end{minipage}
}
\end{figure}
}

\newcommand{\info}[1]{
\encadre{../document_type/images/info.png}{#1}
}

\newcommand{\attention}[1]{
\encadre{../document_type/images/attention.png}{#1}
}

\newcommand{\code}[1]{\texttt{#1}}

\newcommand{\subsubsubsection}[1]{\textbf{#1}
\medskip

}


\newcommand{\titre}{Java Sound}

\newcommand{\diffusion}{\textbf{Libre} & Restreint & Confidentiel}
\newcommand{\etat}{En progression}
\newcommand{\version}{1.0}
\newcommand{\datedoc}{30/01/2006}

\begin{document}

%\frontmatter

%\pagenumbering{Arabic}

\newlength{\larg}
\setlength{\larg}{14.5cm}

\begin{titlepage}

\begin{center}
\includegraphics{../document_type/images/logo_TetraHead_small.png}\\
\end{center}
 
 %\vspace{5.5cm}
 \vspace{\stretch{1}}

{\rule{\larg}{1mm}}\vspace{7mm}
\begin{center}
  {\Huge {\bf {TetraHead - Synth\'etiseur}}} \\
  \medskip
  {\huge \titre}
\end{center}

\vspace{2mm}

{\rule{\larg}{1mm}}
\vspace{2mm} \\
\begin{tabular}{p{11cm} r}
   & {\large \bf 2006}% \\
%   & {\large  \today}
\end{tabular}\\
%\vspace{3cm}
\vspace{\stretch{1}}

\medskip

\begin{center}
\begin{tabular}{|l|l|l|l|}
\hline
\textbf{Diffusion} & \diffusion \\
\hline
�tat & \multicolumn{3}{|l|}{\textbf{\etat}} \\
\hline
Version & \multicolumn{3}{|l|}{\textbf{\version}} \\
\hline
Date & \multicolumn{3}{|l|}{\textbf{\datedoc}} \\
\hline
\end{tabular}

\end{center}

\end{titlepage}



%mainmatter

\begin{historique}
 %\histo{Date}{Pr�nom Nom}{Modification}{Version}
 \histo{30/01/2006}{Vincent MAH�}{Cr�ation}{1.0}
\end{historique}

\section{Description}
Ce document recense les diff�rentes informations relatives � la mise en ouevre du son au sein du langage Java et de son API d�di�e.

\section{Formats de donn�es Audio}

Source : \url{http://java.sun.com/j2se/1.4.2/docs/guide/sound/programmer\_guide/chapter2.html#112348}
\smallskip

In the Java Sound API, a data format is represented by an AudioFormat object, which includes the following attributes:
\begin{itemize}
 \item Encoding technique, usually pulse code modulation (PCM)
 \item Number of channels (1 for mono, 2 for stereo, etc.)
 \item Sample rate (number of samples per second, per channel)
 \item Number of bits per sample (per channel)
 \item Frame rate
 \item Frame size in bytes
 \item Byte order (big-endian or little-endian)
\end{itemize}

\subsection{Encodage}
Seul l'encodage \emph{linear PCM} stocke les valeurs en tant que fonction directe (lin�aire) de l'amplitude sonore. Il doit donc �tre celui utilis�.

\subsection{Canaux}
Nous nous contenterons pour l'instant d'un canal monophonique.

\subsection{�chantillonnage}

Pour l'encodage PCM, une s�quence (frame) contient l'ensemble des valeurs � un instant t dans tous les canaux. Alors, la fr�quence de s�quencement (frame rate) �gale la fr�quence d'�chantillonage, et la taille de la s�quence en octets �gale le nombre de canaux multipli� par la taille de l'�chantillon (taille d'une valeur du son, en octets).

\section{Java Sound API}

Source : \url{http://java.sun.com/j2se/1.4.2/docs/guide/sound/programmer_guide/index.html}


\end{document}
