% Yvonnick Esnault - yvonnick.e@free.fr 
% Gaetan Le Brun - gaetan.lebrun@gmail.com
% Thibaut Leli�vre - thibaut.lelievre@voila.fr
% Vincent Mah� - vmahe@free.fr
% Les chapitres peuvent commencer sur une page paire ou impaire
%openany
\documentclass[a4paper,11pt]{article}
%\frontmatter

\usepackage[T1]{fontenc}
\usepackage{ae,aecompl}
\usepackage[frenchb]{babel}
\usepackage[pdftex]{graphicx}
\usepackage[french]{minitoc}
\usepackage{fancyhdr}
\usepackage{lastpage}
\usepackage{hyperref}
\usepackage{textcomp}
%\usepackage [applemac] {inputenc} % les accents
\usepackage{verbatim}
\usepackage{float}

\evensidemargin=13pt
\oddsidemargin=13pt
\topmargin=-\headheight \advance\topmargin by -\headsep
\textwidth=14.59cm \textheight=21.62cm % normal A4, 1'' margin % 24.62cm
\setlength{\parindent}{8mm} % on remet le retrait de paragraphe

%%% My Vars %%%
\def\mytitle{TetraHead}
\def\mysubject{Synth\'etiseur}
\def\mydate{\today}
\def\myversion{1.0}
\def\me{\small  Yvonnick Esnault - Ga�tan Le Brun - Thibaut Leli�vre - Vincent Mah�}
\def\meandmail{\me \\{\normalsize\tt \mymail}}

%%% Fancy headers to add logo, line \ldots{} %%%
\pagestyle{fancy}
\fancyhf{}

% red�finition du style plain pour les pages sp�ciales (chapter)
\fancypagestyle{plain}{
 \fancyhead{}
 \renewcommand{\headrulewidth}{0pt}
 \renewcommand{\footrulewidth}{0.4pt}
 \lfoot{\textsl{\me\space} \\
\mysubject}
 \cfoot{}
 \rfoot{\textsl{Page \thepage \space sur \pageref{LastPage} }}
}

\renewcommand{\headsep}{50pt}

%\renewcommand{\chaptermark}[1]{\markboth{#1}{}}
\renewcommand{\sectionmark}[1]{\markright{#1}}
%\lhead[\thepage]{\rightmark}
%\rhead[\leftmark]{\thepage}

% Header
%\lhead{\includegraphics[width=2cm]{../document_type/images/logo.png}}
\chead{\mysubject} %\leftmark : contient le nom du chapitre courant.
%\rhead{\includegraphics[width=2cm]{images/logo_itin.png}}

%Footer
\lfoot{\textsl{\me\space}}
\cfoot{}
\rfoot{\textsl{Page \thepage \space sur \pageref{LastPage} }}

%Affiche les ligne en haut, et en bas
\renewcommand{\headrulewidth}{0.4pt}
\renewcommand{\footrulewidth}{0.4pt}

%%% bookmarks, linking in pdf %%%
\hypersetup{pdfauthor={\me},%
            pdftitle={\mytitle},%
            pdfsubject={\mysubject},%
            colorlinks,%
            citecolor=black,%
            filecolor=black,%
            linkcolor=black,%
            urlcolor=black,%
            pdftex}

%%% Commande pour supprimer en-t�tes
%%% et pied de page des pages vides
\newcommand{\clearemptydoublepage}{%
\newpage{\pagestyle{empty}\cleardoublepage}}

%%% Environnement pour le tableau de l'historique
\newenvironment{historique}
{

\begin{center}
{\huge \textbf{Historique}}

\vspace{3cm}
\begin{tabular}{|l|l|l|l|}
\hline
\textbf{Date} & \textbf{Auteur} & \textbf{Modifications} & \textbf{Version} \\
\hline
}
{
\end{tabular}\end{center}
\newpage
}

\newcommand{\histo}[4]
{#1 & #2 & #3 & #4 \\
\hline
}


%\newcommand{\info}[1]{
%\begin{figure}
%  \framebox[\textwidth][l]{
%   \begin{minipage}[c]{0.1\textwidth}
%     \medskip
%
%      \includegraphics{../document_type/images/info.png}
%      \medskip
%
%   \end{minipage} \hfill
%   \begin{minipage}[c]{0.8\textwidth}
%     \medskip
%
%     #1\medskip
%
%   \end{minipage} \hfill
%   \begin{minipage}[c]{0.1\textwidth}
%     \makebox[30pt]{}\par
%   \end{minipage}
%}
%\end{figure}
%}

\newcommand{\encadre}[2]{
\begin{figure}[!h]
  \framebox[\textwidth][l]{
   \begin{minipage}[c]{0.1\textwidth}
     \medskip

      \includegraphics{#1}
      \medskip

   \end{minipage} \hfill
   \begin{minipage}[c]{0.8\textwidth}
     \medskip

     #2\medskip

   \end{minipage} \hfill
   \begin{minipage}[c]{0.1\textwidth}
     \makebox[30pt]{}\par
   \end{minipage}
}
\end{figure}
}

\newcommand{\info}[1]{
\encadre{../document_type/images/info.png}{#1}
}

\newcommand{\attention}[1]{
\encadre{../document_type/images/attention.png}{#1}
}

\newcommand{\code}[1]{\texttt{#1}}

\newcommand{\subsubsubsection}[1]{\textbf{#1}
\medskip

}

\usepackage{url}

% Titre du document qui apparait sur la premi�re page
\newcommand{\titre}{Tests : Modules}

% Les param�tres suivants remplissent le tableau de propri�t�.
% Pour le param�tre \diffusion mettre en gras celui qui nous int�resse
\newcommand{\diffusion}{\textbf{Libre} & Restreint & Confidentiel}
\newcommand{\etat}{En progression}
\newcommand{\version}{1.0}
\newcommand{\datedoc}{24/01/2006}

% D�but du document
\begin{document}

% Cr�e la premi�re page
%\pagenumbering{Arabic}

\newlength{\larg}
\setlength{\larg}{14.5cm}

\begin{titlepage}

\begin{center}
\includegraphics{../document_type/images/logo_TetraHead_small.png}\\
\end{center}
 
 %\vspace{5.5cm}
 \vspace{\stretch{1}}

{\rule{\larg}{1mm}}\vspace{7mm}
\begin{center}
  {\Huge {\bf {TetraHead - Synth\'etiseur}}} \\
  \medskip
  {\huge \titre}
\end{center}

\vspace{2mm}

{\rule{\larg}{1mm}}
\vspace{2mm} \\
\begin{tabular}{p{11cm} r}
   & {\large \bf 2006}% \\
%   & {\large  \today}
\end{tabular}\\
%\vspace{3cm}
\vspace{\stretch{1}}

\medskip

\begin{center}
\begin{tabular}{|l|l|l|l|}
\hline
\textbf{Diffusion} & \diffusion \\
\hline
�tat & \multicolumn{3}{|l|}{\textbf{\etat}} \\
\hline
Version & \multicolumn{3}{|l|}{\textbf{\version}} \\
\hline
Date & \multicolumn{3}{|l|}{\textbf{\datedoc}} \\
\hline
\end{tabular}

\end{center}

\end{titlepage}




% Historique du document. 
\begin{historique}
% Il faut ajouter une balise \histo pour chaque mise � jour
% du document.
% \histo{Date}{Pr�nom Nom}{Modification}{Version}
\histo{30/01/2006}{Vincent MAHE}{Cr�ation}{1.0}
\end{historique}

\section{Description}

Ce document rassemble les compte-rendus de l'ensemble des tests r�alis�s sur la partie \emph{Core.Modules} de l'application.
\smallskip
Un compte-rendu type comporte :
\begin{itemize}
 \item (en titre de section) la personne et la date de la s�rie de tests
 \item la liste des tests effectu�s
 \item le cas �ch�ant, les erreurs corrig�es
\end{itemize}

\section{VM 29/01/2006}

\begin{itemize}
 \item Test du programme TestsCompute, avec un VCO comme exemple
 \item erreurs sur \emph{Module} : les HashTable des ports et des param�tres n'�taient pas initialis�es (code produit par Omondo). Corrig�.
\end{itemize}
\begin{itemize}
 \item Test du module VCO, avec le programme TestsCompute
 \item erreurs sur \emph{VCO} : la fonction d'ondes sinuso�dale rendait 0. Corrig� (division d'\emph{entiers} remplac�e par division � num�rateur \emph{double}).
\end{itemize}

% Fin du document
\end{document}
