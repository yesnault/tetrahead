\section{Installation de \tth}
\label{installation}
\index{Installation}

\subsection{Configuration requise}
\begin{description}
\index{Processeur}
\item[Processeur :] CISC (Intel Pentium\texttrademark, AMD 
Athlon\texttrademark, \ldots) ou RISC (IBM PowerPC\texttrademark, Motorola 
Power G4/5\texttrademark, Sun UltraSparc\texttrademark, \ldots) cadenc� � 2 GHz 
ou plus.
\item[Syst�me d'exploitation :] tout syst�me r�cent pour lequel existe un 
environnement Java 5.0\texttrademark.
\item[Runtime :] \index{Java}Java Runtime Environment (JRE) 5.0\texttrademark
\item[Mat�riel :] carte son reconnue par le syst�me d'exploitation et g�r�e par 
Java 5.0.
\end{description} 

\subsection{Contr�les}

V�rification de la version de java install�e sur la machine

\subsubsection{sur MacOS X}
\index{Installation!MacOS}
\begin{verbatim}
	$ /System/Library/Frameworks/JavaVM.framework/Versions/1.5.0/Commands/java -version
	   java version "1.5.0_05"
	   Java(TM) 2 Runtime Environment, Standard Edition (build 1.5.0_05-83)
	   Java HotSpot(TM) Client VM (build 1.5.0_05-48, mixed mode, sharing)
	$ tar xjf tetrahead.tar.bz2
	$ cd tetrahead
\end{verbatim}

Commande de lancement de \tth � mettre en raccourci :\\
\begin{verbatim}
	$ /System/Library/Frameworks/JavaVM.framework/Versions/1.5.0/Commands/java -jar tetrahead.jar
\end{verbatim} 

\subsubsection{sur Linux}
\index{Installation!Linux}
\begin{verbatim}
	$ /usr/java/jdk1.5.0_03/bin/java -version
	   java version "1.5.0_03"
	   Java(TM) 2 Runtime Environment, Standard Edition (build 1.5.0_03-b07)
	   Java HotSpot(TM) Client VM (build 1.5.0_03-b07, mixed mode, sharing)
	$ tar xjf tetrahead.tar.bz2
	$ cd tetrahead
\end{verbatim} 

Commande de lancement de \tth � mettre en raccourci :
\begin{verbatim}
	$ /usr/java/jdk1.5.0\_03/bin/java -jar tetrahead.jar
\end{verbatim}

\subsubsection{sur Windows}
\index{Installation!Windows}
Ouvrir une console, faire :
\begin{verbatim}
	$ java -version
	   java version "1.5.0_06"
	   Java(TM) 2 Runtime Environment, Standard Edition (build 1.5.0_06-b05)
	   Java HotSpot(TM) Client VM (build 1.5.0_06-b05, mixed mode, sharing)
\end{verbatim} 

D�compresser le fichier \code{tetrahead.tar.bz2} dans le r�pertoire de votre choix.\\
Il s'y cr��e alors un sous-r�pertoire \code{tetrahead}.
\medskip

Pour mettre en raccourci la commande de lancement de \tth, faire un clic droit sur 
le fichier \code{tetrahead.jar}, puis choisir \emph{Cr�er un raccourci}.\\
Faire un double clic sur le raccourci ainsi cr�� pour lancer \tth.

